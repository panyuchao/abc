\section{Introduction}

Protein design is an important method in drug discovery, \xuw{what else??} and
other life science research areas. Due to the long cycle and high cost in
wet-lab experiments, Computational Protein Design (CPD) has become an important
tool for protein engineering~\cite[]{alvizo2007computational}. People have
successfully demonstrated the effectiveness of CPD in applications such as
peptide synthesis \cite[]{ottl1996design}, protein-protein interactions
\cite[]{roberts2012computational}, artificial gene synthesis
\cite[]{villalobos2006gene}.

\xuw{what is the difference/relation between CPD and SCPR?} Specifically, in the
structure-based computational protein design (SCPR) problem, the goal is to
predict amino acid sequences that will fold to a specific protein structure.
More precisely, the aim for CPD is to find the global minimum energy
conformation (GMEC) based on the desired energy function.

The protein design problem has been proven NP-hard \cite[]{pierce2002protein}.
On approach is to obtain approximate solutions, providing no guarantees on the
optimal solution. This problem is modeled as a MAP-MRF inference problem
\cite[]{yanover2006linear}, which can be approximated by a Linear Programming
Relaxation (LPR) problem \cite[]{wainwright2005map}.

On the other hand, there are several methods which can solve the GMEC problem
exactly. Of course, to solve an NP-hard problem exactly, scaling to a larger
protein is the major concern of current these algorithms.

There are two considerations to scale to a larger protein. The first is to
redesign the algorithms to limit the search space as much as possible. Such as
DEE/A* (OSPREY, Donald Lab at DUKE University)~\xuw{used XXX technique to reduce
the search space, while .. (summarize their key ideas)} Branch-and-Bound Search
\cite[]{hong2006protein}, tree decomposition \cite[]{xu2006fast}, AND/OR
Branch-and-Bound \cite[]{marinescu2009and}, Integer Linear Programming
\cite[]{kingsford2005solving}, Cost Network Function \cite[]{traore2013new}.

The second way is to use more machines to perform the computation and tries to
achieve parallel performance. As the search space can be partitioned to allow
each node to handle one partition, we can speed up the search using more
machines. With the development of cloud computing, the cost of computation
resources is going down rapidly, making distributed brutal force solutions more
plausible. Folding@Home~\cite{} is an early attempt to cultivate unused cycles
on desktop PCs to perform protein computation. However, by the definition of
NP-hard, massive computation resources are wasted even for a moderate-sized
problem.

It is difficult to combine the optimized algorithms in a large distributed
computing infrastructure. The main reason is that the optimization part of these
algorithms often requires global states, such as the upper and lower bounds in
the branch-and-bound (BnB) algorithm. Unfortunately, it is a known hard problem
to efficiently sharing such a state in a ``cloud'' system built with commodity
hardware and networking, especially in the presence of node failures.

As the first work to combine a highly optimized GMEC algorithm with massively
scalable cloud-based system, we designed a version of the branch-and-bound (BnB)
search algorithm over a customized Hadoop framework. We demonstrated that the
system scales near-linearly to hundreds of computation servers, attempting
hundreds of billions of braches in parallel, while still takes advantage of all
optimizations in the algorithm.

In our method, the DEE criteria\xuw{need to explain this?} is applied to prune
the infeasible rotamers not only as a pre-filtering algorithm but also in the
branch step. Since the efficiency of the branch-and-bound searching algorithm
heavily depends on the tightness of the bound, we use MPLP
\cite[]{globerson2008fixing} and mini-bucket \cite[]{rollon2010evaluating} to
compute the lower bound, and use Monte Carlo and simulated annealing to find a
good solution as our upper bound. In distributed environment, each node
independently discovers the bound in its part of the search space, and it is
essential to quickly share the global bound to each node. The global bound is
our global state for algorithm optimization.

Our key observation is that the global bound does not affect the correctness of
the algorithm: if a node does not receive the global state, it just uses a
sub-optimal bound and result is no more than performing some unnecessary
computation. Thus we can share the bound in a best-effort way and tolerate the
occasional inconsistency of the states. As we will show in the paper, the extra
overhead is negligible and the gain of performance is significant comparing to
the brutal force approach.

In this paper, we demonstrate two approaches to perform the best-effort state
sharing. The first approach uses probabilistic data propagation, resulting in an
algorithm that runs on unmodified MapReduce~\cite[]{} framework. The second
approach is based on asynchronous communication and eventual consistency
model~\cite[]{}, which requires minor modifications of the Hadoop framework, but
can improve performance by XX%.

Our system inherits all fault-tolerance benefits of the cloud-based systems. In
one of our experiments, we killed XXX computation nodes and the analysis still
completed normally. This feature is the key to make the cloud economics model
work: researchers can take advantage of the off-peak machine time to perform the
GMEC tasks and can kill some tasks whenever there is better use of the machines
– the killed tasks will automatically restart on other machines without
affecting the final results. The cost of off-peak machine times (such as the
“spot instances” in Amazon’s EC2 cloud computing platform~\cite[]{}
http://aws.amazon.com/ec2/purchasing-options/spot-instances/ ) is only a
fraction of the normal price, making our approach financially practical.

We have three contributions in this paper

\begin{enumerate}
  \item We integrated several optimizations to improve the BnB algorithm. \xuw{key
ideas of the algorithms}

  \item We demonstrated that by relaxing the consistency requirements of the global
bound in the BnB algorithm, we create a massively scalable computation system
that runs the highly optimized version of BnB algorithm to solve GMEC problems
in protein design. The system runs efficiently on commodity cloud computing
environment and tolerates failures effectively.

  \item We obtain the optimal results on XX protein design problems, which was not
possible using even the state-of-the-art single machine algorithms. \xuw{more comp-bio contributions here?}

\end{enumerate}

We want to emphasis that although this paper focuses on branch-and-bound
algorithm, our methodology of handling global states by relaxing the consistency
requirements can be applied to many different search algorithms of the same
nature.

The remaining of the paper goes as follows..
