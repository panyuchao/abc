\section{Introduction}
In the past few years, Computational Protein Design (CPD) has become an important tool for protein engineering \cite[]{alvizo2007computational}, such as peptide synthesis \cite[]{ottl1996design}, protein-protein interactions \cite[]{roberts2012computational}, artificial gene synthesis \cite[]{villalobos2006gene}, etc. In the structure-based computational protein design problem, the goal is to predict amino acid sequences that will fold to a specific protein structure. More precisely, the aim for CPD is to find the global minimum energy conformation (GMEC) based on the desired energy function.

The protein design problem has been proven NP-hard \cite[]{pierce2002protein}. This problem is modeled as a MAP-MRF inference problem \cite[]{yanover2006linear}, which can be approximated by a Linear Programming Relaxation (LPR) problem \cite[]{wainwright2005map}. On the other hand, there exists several methods which can solve the GMEC problem exactly, such as DEE/A* (OSPREY, Donald Lab at DUKE University), Branch-and-Bound Search \cite[]{hong2006protein}, tree decomposition \cite[]{xu2006fast}, AND/OR Branch-and-Bound \cite[]{marinescu2009and}, Integer Linear Programming \cite[]{kingsford2005solving}, Cost Network Function \cite[]{traore2013new}.

Our work is the first attempt to apply the branch-and-bound (BnB) search algorithm on distributed platform (such as MapReduce) to solve the Computational Protein Design problem. In our method, the DEE criteria is applied to prune the infeasible rotamers not only as a pre-filtering algorithm but also in the branch step. Since the efficiency of the branch-and-bound searching algorithm heavily depends on the tightness of the bound, we use MPLP \cite[]{globerson2008fixing} and mini-bucket \cite[]{rollon2010evaluating} to compute the lower bound, and use Monte Carlo and simulated annealing to find a good solution as our upper bound. 